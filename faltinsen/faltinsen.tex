\documentclass{article}
\usepackage{mathtools, hyperref, amssymb}
\usepackage{bm}

\newcommand{\R}{\mathbb{R}}
\newcommand{\peakFreq}{\omega}

\newcommand{\Hs}{H_{\frac{1}{3}}}
\newcommand{\Hsj}{\Hs^{(j)}}
\newcommand{\mZeroj}{m_0^{(j)}}
\newcommand{\significantHeightComponent}[1]{\significantHeightSymbol_{#1}}
\newcommand{\significantHi}[1]{\significantHeightComponent{#1}}
\newcommand{\significantHj}{\significantHi{j}}
\newcommand{\significantH}{\mathbf{\significantHeightSymbol}}
\newcommand{\significantHvec}{\langle \significantHj \rangle_{j \in \Sigma}}

\newcommand{\significantFreqSymbol}{\hat{\omega}}
\newcommand{\significantFreqComponent}[1]{\significantFreqSymbol_{#1}}
\newcommand{\significantWi}[1]{\significantFreqComponent{#1}}
\newcommand{\significantWj}{\significantWi{j}}
\newcommand{\significantW}{\bm{\significantFreqSymbol}}
\newcommand{\significantWvec}{\langle \significantWj \rangle_{j \in \Sigma}}

\newcommand{\significantDirectionSymbol}{\bar{\theta}}
\newcommand{\significantDirectionJ}{\significantDirectionSymbol_j}
\newcommand{\significantDirection}{\bm{\significantDirectionSymbol}}

\newcommand{\ochihubbleVector}{\bm{S}_\eta}
\newcommand{\spreadingVector}{\bm{D}}

\newcommand{\formParamSymbol}{\lambda}
\newcommand{\li}[1]{\formParamSymbol_{#1}}
\newcommand{\lj}{\li{j}}
\newcommand{\formParam}{\bm{\formParamSymbol}}

\newcommand{\angularWidthDegreesJ}{\sigma_{deg, j}}
\newcommand{\angularWidthRadiansJ}{\sigma_j}

\newcommand{\rao}{\bm{A}_{\text{RAO}}}

\begin{document}

\section{Introduction}
This document is a supplementary to Faltinsen's book on sea loads
~\cite{faltinsen90}. This book is not always very verbose, and often
just states a result or formula, and perhaps gives some hints to the
deduction of it. This document is meant as a placeholder for the
reader's own deductions or other further investigations of the text.

\section{Wave simulation}
The standard way of defining the amplitude components from the wave
spectrum is in equation 2.23:

\begin{equation}
  \frac{1}{2}A_j = S(\omega_j)\Delta\omega
\end{equation}

What does this mean? Well, the left hand side is the variance of the
$j$th wave component with respect of time (the deduction is trivial
now, but I should probably write it down so that I'll remember in the
future). The right hand side is an approximation of it, given that $S$
is defined as (citation needed)

\begin{equation}
  S(\omega) = \frac{d(\sigma^2)}{d\omega}.
\end{equation}

Hence

\begin{equation}
  \sigma^2 = \int_{0}^{\infty}S(\omega)d\omega.
\end{equation}

Discretizing the frequency domain, let $\omega_k$ be the sample
frequency of the interval $[\omega_k - a_k, \omega_k + b_k), a_k, b_k >
  0$. Then the variance contribution of this frequency is

\begin{equation}
  \sigma_k^2 = \int_{\omega_k - a_k}^{\omega_k + b_k}S(\omega)d\omega
\end{equation}

Note that if we partition the frequency domain such that $\omega_k +
b_k = \omega_{k+1} - a_{k+1}, k = 1, \ldots N$, and $S(\omega) = 0,
\omega \notin [\omega_1 - a_1, \omega_N + b_N )$ we get

\begin{equation}
  \sigma^2 = \int_{0}^{\infty}S(\omega)d\omega
  = \sum\limits_{k=1}^{N}\int_{\omega_k - a_k}^{\omega_k + b_k}S(\omega)d\omega
  = \sum\limits_{k=1}^{N}\sigma_k^2
\end{equation}
\noindent
which seems reasonable.

For uniform grids with grid spacing $\Delta\omega$, we have $a_k = b_k =
\Delta\omega/2$, so that

\begin{equation}
  \sigma_k^2 = \int_{\omega_k - \Delta\omega/2}^{\omega_k + \Delta\omega/2}S(\omega)d\omega
\end{equation}

and so, by applying the midpoint rule
(\url{http://en.wikipedia.org/wiki/Rectangle_method}), we arrive at

\begin{equation}
  \label{eq:sigmaapprox}
  \sigma_k^2 \approx S(\omega_k)\Delta\omega
\end{equation}

Now, Faltinsen recommends randomly to shift the frequencies of the
wave components. This results in a non-uniform grid, and the
approximation in equation~\ref{eq:sigmaapprox} gets even more
inaccurate. Is this method sound? For big $N$s it \emph{is} of course. But
could we reduce the number of wave components by applying another
integration approximation method like, say the trapezoid method? The
calculation of the amplitudes would require some more time, but the
curve generation time would be reduced.

\section{Long Term Sea States}

\subsection{Formula 2.39}

This equation expresses the probability that

\begin{equation}
  \label{eq:2.39}
  P(H) = 1 - \sum\limits_{j = 1}^M e^{(-2H^2/(\Hsj)^2)} p_j,
\end{equation}

\noindent
where $P(H)$ is the probability that the crest-to-trough height does
not exceeed $H$, $j \in {1, ..., M}$ is one of $M$ sea states, $p_j$
is the probability the sea to be in sea state $j$, and $\Hsj$ is the
significant wave height for sea state $j$.

Now, this equation is derived from equation 2.32 in ~\cite{faltinsen90}:

\begin{equation}
  \label{eq:2.32}
  p(A) = \frac{A}{m_0} e^{-A^2/(2m_0)},
\end{equation}

\noindent
where $m_0$ is the zeroeth moment of the wave spectrum $S(\omega)$:
\begin{equation}
  \label{eq:2.23below}
  m_k = \int_0^\infty \omega^k S(\omega) d\omega,
\end{equation}

\noindent
hence $m_0 = \int_0^\infty S(\omega) d\omega = \sigma^2$, i.e. the
variance of the wave height distribution, $N(0, \sigma)$ (p. 23).

To derive equation~\ref{eq:2.32} we note that the cumulative Rayleigh
distribution function is

\begin{equation}
  \label{eq:rayleighCumulative}
  p(x \leq A) = 1 - e^{-A^2/(2m_0)}.
\end{equation}

\noindent
and that the significant wave height is often set as
\begin{equation}
  \label{eq:significantWaweHeight}
  \Hs = 4\sqrt{m_0} = 4\sigma.
\end{equation}

\noindent
Hence, $m_0 = \Hs^2/16$.

Now, let $\varsigma$ denote the current sea state, and let $\mZeroj$ and
$\Hsj$ be the zeroeth moment and significant wave height,
respectively, for some sea state $j$. Setting $A = H/2$, we then get
\begin{eqnarray*}
  p(x \leq A | \varsigma = j)
  &=& p(x \leq H/2) \\
  &=& 1 - e^{-A^2/(2\mZeroj)} \\
  &=& 1 - e^{-(H/2)^{2} / 2({\Hsj}^2/16)} \\
  &=& 1 - e^{-2H^2/{\Hsj}^2}.
\end{eqnarray*}

\noindent
Finally, denoting the probility for being in sea state $j$ as
$p(\varsigma = j) = p_j$, and assuming being in two different sea
states are two disjunct and independent events, we get
\begin{eqnarray*}
  P(H)
  &=& p(x \leq H) \\
  &=& 1 - p(x > H) \\
  &=& 1 - \sum\limits_{j = 1}^M p(x > H | \varsigma = j)p(\varsigma = j) \\
  &=& 1 - \sum\limits_{j = 1}^M e^{-2H^2/(\Hsj)^2}p_j
\end{eqnarray*}
\noindent $\square$

\bibliography{faltinsen}{}
\bibliographystyle{plain}

\end{document}
